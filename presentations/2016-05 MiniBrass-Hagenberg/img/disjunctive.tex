\begin{frame}[fragile]{Non-overlaps (Manually)}

\tikzset{
    task/.style={rounded corners, minimum height=3.8ex,rectangle,anchor=south west},
    task1/.style={task,draw=black, top color=isseorange!5, bottom color=isseorange!30},    
    task2/.style={task,draw=black, top color=issegrey!5, bottom color=issegrey!30},
    task3/.style={task,draw=black, top color=BrickRed!5, bottom color=BrickRed!30},
}
 
%\vspace*{-6ex}
\begin{figure}
\begin{tikzpicture}[scale=1.0]
    % Draw axes
    \draw [<->,thick] (0,2) node (yaxis) [above] {}
        |- (4.0,0) node (xaxis) [right] {$t$};    

    \draw [<->,thick] (6,2) node (yaxis) [above] {}
        |- (10.0,0) node (xaxis) [right] {$t$};      
% first slide

    \node[task1] at (1,0.5) { cut onions };        
    \node[task2] at (1.5,1.1) { cook onions };         

    \node[task1] at (6,0.5) { cut onions };        
    \node[task2] at (8,1.1) { cook onions };          
 
  %  \node[task3] at (2,1.5) { cook spaghetti };  

	\node at (1.9, 2.6) {\color{BrickRed} \Huge \Lightning };
	\node at (7.9, 2.6) {\color{ForestGreen} \Huge \checkmark };
	
	\node[text centered, anchor=north] at (1,0) { 1 }; \draw[thick] (1,0.05) -- (1,-.05);
	\node[text centered, anchor=north] at (2,0) { 2 }; \draw[thick] (2,0.05) -- (2,-.05);
	\node[text centered, anchor=north] at (3,0) { 3 }; \draw[thick] (3,0.05) -- (3,-.05);	
%	\node[text centered, anchor=north] at (4,0) { 4 }; \draw[thick] (4,0.05) -- (4,-.05);
	\node[text centered, anchor=north] at (7,0) { 1 }; \draw[thick] (7,0.05) -- (7,-.05);	
	\node[text centered, anchor=north] at (8,0) { 2 }; \draw[thick] (8,0.05) -- (8,-.05);
	\node[text centered, anchor=north] at (9,0) { 3 }; \draw[thick] (9,0.05) -- (9,-.05);
	   
\end{tikzpicture}
\end{figure}

\begin{lstlisting}
predicate nonoverlap(var TASK: task1, var TASK: task2) = 
             s[task1] + d[task1] <= s[task2] \/ 
             s[task2] + d[task2] <= s[task1];
             
             
constraint nonoverlap(cut_on, cook_on);
\end{lstlisting}
\end{frame}

\begin{frame}[fragile]{Non-overlaps (Global Constraint)}

\tikzset{
    task/.style={rounded corners, minimum height=3.8ex,rectangle,anchor=south west},
    task1/.style={task,draw=black, top color=isseorange!5, bottom color=isseorange!30},    
    task2/.style={task,draw=black, top color=issegrey!5, bottom color=issegrey!30},
    task3/.style={task,draw=black, top color=BrickRed!5, bottom color=BrickRed!30},
}
 
%\vspace*{-6ex}
\begin{figure}
\begin{tikzpicture}[scale=1.0]
    % Draw axes
    \draw [<->,thick] (0,2) node (yaxis) [above] {}
        |- (4.0,0) node (xaxis) [right] {$t$};    

    \draw [<->,thick] (6,2) node (yaxis) [above] {}
        |- (10.0,0) node (xaxis) [right] {$t$};      
% first slide

    \node[task1] at (1,0.5) { cut onions };        
    \node[task2] at (1.5,1.1) { cook onions };         

    \node[task1] at (6,0.5) { cut onions };        
    \node[task2] at (8,1.1) { cook onions };          
 
  %  \node[task3] at (2,1.5) { cook spaghetti };  

	\node at (1.9, 2.6) {\color{BrickRed} \Huge \Lightning };
	\node at (7.9, 2.6) {\color{ForestGreen} \Huge \checkmark };
	
	\node[text centered, anchor=north] at (1,0) { 1 }; \draw[thick] (1,0.05) -- (1,-.05);
	\node[text centered, anchor=north] at (2,0) { 2 }; \draw[thick] (2,0.05) -- (2,-.05);
	\node[text centered, anchor=north] at (3,0) { 3 }; \draw[thick] (3,0.05) -- (3,-.05);	
%	\node[text centered, anchor=north] at (4,0) { 4 }; \draw[thick] (4,0.05) -- (4,-.05);
	\node[text centered, anchor=north] at (7,0) { 1 }; \draw[thick] (7,0.05) -- (7,-.05);	
	\node[text centered, anchor=north] at (8,0) { 2 }; \draw[thick] (8,0.05) -- (8,-.05);
	\node[text centered, anchor=north] at (9,0) { 3 }; \draw[thick] (9,0.05) -- (9,-.05);
	   
\end{tikzpicture}
\end{figure}

\begin{lstlisting}
include "disjunctive.mzn"; % global definition from standard library
predicate disjunctive(array [int] of var int: s,
                      array [int] of var int: d)
             
constraint nonoverlap( [ s[cut_on], s[cook_on] ], 
                          [ du[cut_on], du[cook_on] ]);
\end{lstlisting}
\end{frame}
