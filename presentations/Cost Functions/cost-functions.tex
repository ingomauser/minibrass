%\documentclass[10pt,xcolor={dvipsnames},fleqn]{beamer}
\documentclass[handout,10pt,xcolor={dvipsnames},fleqn]{beamer}
\usepackage{isse}


\usepackage{apalike}
\usepackage[utf8]{inputenc}
\usepackage{pdfpages}
%\usepackage{ngerman}
\usepackage{stmaryrd,amsmath,amssymb}
\usepackage{color}
\usepackage{enumerate}
\usepackage[makeroom]{cancel}
\usepackage{mdframed}
\usepackage{xskak}

\usepackage{fancyvrb}
\usepackage{fixltx2e}
%\usepackage{bera}

\usepackage{marvosym}
\setchessboard{
showmover=false}
\usepackage[noend]{algpseudocode}   % package for algorithms
\usepackage{algorithm}
\usepackage{tikz}

\usepackage[absolute,overlay]{textpos}

\usetikzlibrary{trees,calc,shapes,arrows,matrix,shadows,decorations.markings}

\mdfdefinestyle{theoremstyle}{
linecolor=red,linewidth=2pt,
frametitlerule=true,
frametitlebackgroundcolor=gray!20,
innertopmargin=\topskip,
}
\definecolor{LRed}{rgb}{1,.8,.8}
\definecolor{MRed}{rgb}{1,.6,.6}
\definecolor{HRed}{rgb}{1,.2,.2}

\usepackage{listings}
\lstdefinelanguage{mzn}
{
	morekeywords={var,int,solve,bool,not,search,satisfy,endif,maximize,minimize,float,constraint,sum,forall,exists,array,of,include,predicate,then,commit,post,set,function,if,else,repeat,next,ann,break},
	sensitive=false,
	morecomment=[l]{\%},
	morecomment=[s]{/*}{*/},
	morestring=[b]",
}

\definecolor{lightlightgray}{gray}{0.95}
\definecolor{forestgreen}{HTML}{009B55}
\definecolor{thermicred}{rgb}{0.82, 0.1, 0.26}
\lstset
{
	basicstyle=\ttfamily\small,
	commentstyle=\ttfamily\color{thermicred},
	stringstyle=\ttfamily\color{isseorange},
	keywordstyle=\ttfamily\color{blue},
	tabsize=2,
	showstringspaces=false,
	flexiblecolumns=true,
	captionpos=b,	
	backgroundcolor=\color{lightlightgray},
	frame=single,
	 xleftmargin=\parindent,
}

\lstset{language=mzn}
\interfootnotelinepenalty=10000

% ====== custom commands

\newcommand{\prosumer}[1]{\ensuremath{\mathtt{#1}}}
% Soft Constraint Example
\newcommand{\constraintName}[1]{\ensuremath{\mathtt{#1}}}
% Biogas Constraints
\newcommand{\biogas}{biogas}
\newcommand{\biogasShort}{bio}
\newcommand{\gasFull}{\ensuremath{\constraintName{gasFull}_\mathtt{\biogasShort}}}
\newcommand{\ecoSweet}{\ensuremath{\constraintName{ecoSweet}_\mathtt{\biogasShort}}}
\newcommand{\onOff}{\ensuremath{\constraintName{onOff}_\mathtt{\biogasShort}}}
% Thermal Plant Constraints
\newcommand{\thermal}{thermal}
\newcommand{\thermalShort}{therm}
\newcommand{\ecoOpt}{\ensuremath{\constraintName{ecoOpt}_\mathtt{\thermalShort}}}
\newcommand{\inertia}{\ensuremath{\constraintName{inertia}_\mathtt{\thermalShort}}}
\newcommand{\ecoGood}{\ensuremath{\constraintName{ecoGood}_\mathtt{\thermalShort}}}
\newcommand{\hLevelThermal}[1]{$H_#1^\mathtt{\thermalShort}$}
% Electric Vehicle
\newcommand{\ev}{EV}
\newcommand{\limitBatteryUsage}{\ensuremath{\constraintName{limitBU}_\mathtt{\ev}}}
\newcommand{\prefBatteryLevel}{\ensuremath{\constraintName{prefBL}_\mathtt{\ev}}}
\newcommand{\earlyBird}{\ensuremath{\constraintName{earlyBird}_\mathtt{\ev}}}
% Organization
\newcommand{\org}{org}
\newcommand{\minMaxViolation}{\ensuremath{\constraintName{violation}_\mathtt{\org}}}
\newcommand{\hLevelOrg}[1]{$H_#1^\mathtt{\org}$}

\newcommand{\Variable}{X}
\newcommand{\LocalVariable}{\widehat{\Variable}}
\newcommand{\Domain}{D}
\newcommand{\Constraint}{C}
\newcommand{\ConstraintRelationship}{\mathcal{R}}

\newcommand{\valuation}{v}
\newcommand{\constraint}[1]{\mathrm{#1}}

\newcommand{\plantconstraint}[3]{  
\ifx#1b \constraint{best}[#3]
\else \ifx#1g \constraint{good}[#3]
\else \ifx#1a \constraint{acc}[#3]
\else \ifx#1d \constraint{diff}
\else \ifx#1l \constraint{low}[#3]
\else \ifx#1h \constraint{high}[#3]
\else \ifx#1o \constraint{org}[#3]
   \else
   \constraint{#1}_{#2}^{#3} 
   
   
\fi \fi \fi \fi \fi \fi \fi}
\usepackage{stmaryrd}

\newcommand{\code}[1]{\normalfont\texttt{\spaceskip=3pt\frenchspacing\def\{{\char123}\def\}{\char125}\def\^{\char94}\def\_{\char95}#1}}
\newcommand{\varit}[1]{{\frenchspacing\ensuremath{\normalfont\textsl{#1}}}}
\newcommand{\macit}[1]{{\frenchspacing\ensuremath{\normalfont\textsf{#1}}}}
\newcommand{\Eta}{\mathrm{H}}
\newcommand{\Mu}{\mathrm{M}}
\newcommand{\Nu}{\mathrm{N}}

\newcommand{\NZ}{\mathbb{N}}
\newcommand{\RZ}{\mathbb{R}}
\newcommand{\RZp}{\RZ_{\geq 0}}
\newcommand{\powerset}{\mathcal{P}}
\newcommand{\limp}{\mathrel{\Rightarrow}}
\newcommand{\compfun}{\mathbin{\circ}}
\newcommand{\isorel}{\mathrel{\cong}}
\newcommand{\restrict}[2]{{#1}\mathnormal{\upharpoonright}{#2}}
\newcommand{\natto}{\mathrel{\dot{\mathnormal{\to}}}}
\let\lbagold\lbag
\let\rbagold\rbag
\def\lbag{\mathopen{\lbagold}}
\def\rbag{\mathclose{\rbagold}}

\DeclareMathOperator{\Minop}{\mathrm{Min}}
\newcommand{\Min}[1]{\Minop^{#1}}
\DeclareMathOperator{\Maxop}{\mathrm{Max}}
\newcommand{\Max}[1]{\Maxop^{#1}}
\DeclareMathOperator{\finsets}{\mathcal{P}_{\mathrm{fin}}}
\DeclareMathOperator{\nefinsets}{\mathcal{P}_{\mathrm{fin}^+}}
%\DeclareMathOperator{\incfinsets}{\mathcal{I}_{\mathrm{fin}}}
\newcommand{\incfinsets}[1]{\mathcal{I}_{\mathrm{fin}}^{#1}}
\newcommand{\lowersubseteq}[1]{\mathrel{\subseteq_{#1}}}
\newcommand{\lowersupseteq}[1]{\mathrel{\supseteq_{#1}}}
\newcommand{\lowersubset}[1]{\mathrel{\subset_{#1}}}
\newcommand{\lowersupset}[1]{\mathrel{\supset_{#1}}}
\newcommand{\uppersubseteq}[1]{\mathrel{\subseteq^{#1}}}
\newcommand{\uppersupseteq}[1]{\mathrel{\supseteq^{#1}}}
\newcommand{\uppersubset}[1]{\mathrel{\subset^{#1}}}
\newcommand{\uppersupset}[1]{\mathrel{\supset^{#1}}}
\newcommand{\lowercup}[1]{\mathbin{\cup_{#1}}}
\newcommand{\uppercup}[1]{\mathbin{\cup^{#1}}}

\DeclareMathOperator{\finmsets}{\mathcal{M}_{\mathrm{fin}}}
\DeclareMathOperator{\nefinmsets}{\mathcal{M}_{\mathrm{fin}^+}}
\newcommand{\mcup}{\mathbin{\mathnormal{\cup}\llap{\text{\fontsize{8pt}{8pt}\selectfont$-$}}}}
\newcommand{\submseteq}{%
\mathrel{\mathchoice%
{\mathnormal{\subseteq}\llap{\text{\raisebox{0.3pt}{\fontsize{8pt}{8pt}\selectfont\rotatebox{90}{$-$}\hspace{1.8pt}}}}}%
{\mathnormal{\subseteq}\llap{\text{\raisebox{0.3pt}{\fontsize{8pt}{8pt}\selectfont\rotatebox{90}{$-$}\hspace{1.8pt}}}}}%
{\mathnormal{\subseteq}\llap{\text{\raisebox{-0.3pt}{\fontsize{5pt}{5pt}\selectfont\rotatebox{90}{$-$}\hspace{1.4pt}}}}}%
{\mathnormal{\subseteq}\llap{\text{\raisebox{-0.3pt}{\fontsize{5pt}{5pt}\selectfont\rotatebox{90}{$-$}\hspace{1.4pt}}}}}%
}}
\newcommand{\supmseteq}{\mathrel{\reflectbox{$\submseteq$}}}
\newcommand{\lowersubmseteq}[1]{\mathrel{\submseteq_{#1}}}
\newcommand{\uppersubmseteq}[1]{\mathrel{\submseteq^{#1}}}
\newcommand{\submset}{%
\mathrel{\mathchoice%
{\mathnormal{\subset}\llap{\text{\raisebox{-0.8pt}{\fontsize{8pt}{8pt}\selectfont\rotatebox{90}{$-$}\hspace{1.8pt}}}}}%
{\mathnormal{\subset}\llap{\text{\raisebox{-0.8pt}{\fontsize{8pt}{8pt}\selectfont\rotatebox{90}{$-$}\hspace{1.8pt}}}}}%
{\mathnormal{\subset}\llap{\text{\raisebox{-0.3pt}{\fontsize{7pt}{7pt}\selectfont\rotatebox{90}{$-$}\hspace{1pt}}}}}%
{\mathnormal{\subset}\llap{\text{\raisebox{-0.3pt}{\fontsize{7pt}{7pt}\selectfont\rotatebox{90}{$-$}\hspace{1pt}}}}}%
}}
\newcommand{\supmset}{\mathrel{\reflectbox{$\submset$}}}
\newcommand{\lowersubmset}[1]{\mathrel{\submset_{#1}}}
\newcommand{\uppersubmset}[1]{\mathrel{\submset^{#1}}}

\DeclareMathOperator{\collapseset}{\mathcal{C}}

\newcommand{\category}[1]{\mathrm{#1}}
\newcommand{\POcat}{\category{PO}}
\newcommand{\uSLcat}{\category{uSL}}
\newcommand{\poMoncat}{\category{poMon}}
\newcommand{\jMoncat}{\category{jMon}}
\newcommand{\mMoncat}{\category{mMon}}
\newcommand{\xMoncat}{{x}\category{Mon}}
\newcommand{\PVScat}{\category{PVS}}
\newcommand{\cSRngcat}{\category{cSRng}}
\newcommand{\DAGcat}{\category{DAG}}

\newcommand{\idfun}[1]{1_{#1}}
\newcommand{\functor}[1]{\mathit{#1}}
\DeclareMathOperator{\POfun}{\functor{PO}}
\DeclareMathOperator{\uSLfun}{\functor{uSL}}
\DeclareMathOperator{\poMonfun}{\functor{poMon}}
\DeclareMathOperator{\jMonfun}{\functor{jMon}}
\DeclareMathOperator{\mMonfun}{\functor{mMon}}
\DeclareMathOperator{\xMonfun}{\text{$x$}\functor{Mon}}
\DeclareMathOperator{\PVSfun}{\functor{PVS}}
\DeclareMathOperator{\cSRngfun}{\functor{cSRng}}
\DeclareMathOperator{\DAGfun}{\functor{DAG}}

\newcommand{\uSLfree}[1]{\uSLfun\langle#1\rangle}
\newcommand{\uSLeta}{\eta^{\uSLcat}}
\newcommand{\uSLetaat}[1]{\uSLeta_{#1}}
\newcommand{\uSLlift}[1]{{#1}^{\sharp_{\uSLcat}}}

\newcommand{\poMonfree}[1]{\poMonfun\langle#1\rangle}
\newcommand{\poMoneta}{\eta^{\poMoncat}}
\newcommand{\poMonetaat}[1]{\poMoneta_{#1}}
\newcommand{\poMonlift}[1]{{#1}^{\sharp_{\poMoncat}}}

\newcommand{\jMonfree}[1]{\jMonfun\langle#1\rangle}
\newcommand{\jMoneta}{\eta^{\jMoncat}}
\newcommand{\jMonetaat}[1]{\jMoneta_{#1}}
\newcommand{\jMonlift}[1]{{#1}^{\sharp_{\jMoncat}}}

\newcommand{\mMonfree}[1]{\mMonfun\langle#1\rangle}
\newcommand{\mMoneta}{\eta^{\mMoncat}}
\newcommand{\mMonetaat}[1]{\mMoneta_{#1}}
\newcommand{\mMonlift}[1]{{#1}^{\sharp_{\mMoncat}}}

\newcommand{\PVSfree}[1]{\PVSfun\langle#1\rangle}
\newcommand{\PVSeta}{\eta^{\PVScat}}
\newcommand{\PVSetaat}[1]{\PVSeta_{#1}}
\newcommand{\PVSlift}[1]{{#1}^{\sharp_{\PVScat}}}

\newcommand{\xMonfree}[1]{\xMonfun\langle#1\rangle}
\newcommand{\xMoneta}{\eta^{\xMoncat}}
\newcommand{\xMonetaat}[1]{\xMoneta_{#1}}
\newcommand{\xMonlift}[1]{{#1}^{\sharp_{\xMoncat}}}

\newcommand{\cSRngfree}[1]{\cSRngfun\langle#1\rangle}
\newcommand{\cSRngeta}{\eta^{\cSRngcat}}
\newcommand{\cSRngetaat}[1]{\cSRngeta_{#1}}
\newcommand{\cSRnglift}[1]{{#1}^{\sharp_{\cSRngcat}}}

\newcommand{\POfree}[1]{\POfun\langle#1\rangle}
\newcommand{\POeta}{\eta^{\POcat}}
\newcommand{\POetaat}[1]{\POeta_{#1}}
\newcommand{\POlift}[1]{{#1}^{\sharp_{\POcat}}}

\newcommand{\mtimes}[1]{\mathbin{\tilde{\cdot}_{#1}}}
\newcommand{\mplus}[1]{\mathbin{\tilde{\cup}_{#1}}}
\newcommand{\ftimes}[1]{\mathbin{\tilde{\mcup}^{#1}}}
\newcommand{\fplus}[1]{\mathbin{\tilde{\cup}_{#1}}}

\DeclareMathOperator{\scope}{\mathrm{sc}}
\DeclareMathOperator{\defdom}{\mathrm{def}}

\newcommand{\reflclos}[1]{\mathrel{(#1)^=}}
\newcommand{\transclos}[2][+]{\mathrel{(#2)^{#1}}}
\newcommand{\refltransclos}[1]{\mathrel{(#1)^*}}

\newcommand{\XPDrel}[2][\pi]{\rightsquigarrow^{#1}_{#2}}
\newcommand{\XPDreleq}[2][\pi]{\rightsquigarrow^{#1, =}_{#2}}
\newcommand{\XPDord}[2][\pi]{<^{#1}_{#2}}
\newcommand{\XPDordeq}[2][\pi]{\geq^{#1}_{#2}}
\newcommand{\XPDleq}[2][\pi]{\leq^{#1}_{#2}}
\newcommand{\XPDgeq}[2][\pi]{\geq^{#1}_{#2}}
\newcommand{\XPDw}[2][\pi]{w^{#1}_{#2}}
\newcommand{\XPDW}[2][\pi]{W^{#1}_{#2}}
\newcommand{\XPDk}[2][\pi]{k^{#1}_{#2}}

\newcommand{\SPDrel}{\XPDrel[\mathrm{SPD}]}
\newcommand{\SPDreleq}{\XPDreleq[\mathrm{SPD}]}
\newcommand{\SPDleq}{\XPDleq[\mathrm{SPD}]}
\newcommand{\SPDgeq}{\XPDgeq[\mathrm{SPD}]}
\newcommand{\SPDord}{\XPDord[\mathrm{SPD}]}
\newcommand{\SPDw}{\XPDw[\mathrm{SPD}]}
\newcommand{\SPDW}{\XPDW[\mathrm{SPD}]}
\newcommand{\DPDrel}{\XPDrel[\mathrm{DPD}]}
\newcommand{\DPDreleq}{\XPDreleq[\mathrm{DPD}]}
\newcommand{\DPDord}{\XPDord[\mathrm{DPD}]}
\newcommand{\DPDw}{\XPDw[\mathrm{DPD}]}
\newcommand{\DPDW}{\XPDW[\mathrm{DPD}]}
\newcommand{\TPDrel}{\XPDrel[\mathrm{TPD}]}
\newcommand{\TPDreleq}{\XPDreleq[\mathrm{TPD}]}
\newcommand{\TPDleq}{\XPDleq[\mathrm{TPD}]}
\newcommand{\TPDgeq}{\XPDgeq[\mathrm{TPD}]}
\newcommand{\TPDord}{\XPDord[\mathrm{TPD}]}
\newcommand{\TPDw}{\XPDw[\mathrm{TPD}]}
\newcommand{\TPDW}{\XPDW[\mathrm{TPD}]}

\DeclareMathSymbol{\UPi}{\mathalpha}{operators}{"05}



\renewcommand{\submseteq}{%
\mathrel{\mathchoice%
{\mathnormal{\subseteq}\llap{\text{\raisebox{0.0pt}{\fontsize{7.5pt}{7.5pt}\selectfont\rotatebox{90}{$-$}\hspace{1.6pt}}}}}%
{\mathnormal{\subseteq}\llap{\text{\raisebox{0.0pt}{\fontsize{7.5pt}{7.5pt}\selectfont\rotatebox{90}{$-$}\hspace{1.6pt}}}}}%
{\mathnormal{\subseteq}\llap{\text{\raisebox{-0.3pt}{\fontsize{7pt}{7pt}\selectfont\rotatebox{90}{$-$}\hspace{1pt}}}}}%
{\mathnormal{\subseteq}\llap{\text{\raisebox{-0.3pt}{\fontsize{7pt}{7pt}\selectfont\rotatebox{90}{$-$}\hspace{1pt}}}}}%
}}


\tikzset{
   main node/.style={circle,fill=black!15,draw,font=\sffamily},
   constraint node/.style={main node, circle, inner sep=2pt,font=\sffamily\small},   
   treestyle/.style={rectangle,fill=black!15,draw,font=\sffamily}
}


\mdtheorem[style=theoremstyle]{definition}{Definition}

\renewcommand{\vec}[1]{\mathbf{#1}}
\newcommand{\tupleOf}[1]{\langle #1 \rangle}
\newcommand{\cemph}[1]{\alert{#1}}
\usepackage{framed}
\usepackage{ifthen}

\usetikzlibrary{decorations.pathmorphing,calc,shadows.blur,shadings}
\usetikzlibrary{mindmap,trees,automata,arrows}
\usepackage{extrabeamercmds}

\newcommand{\hFirst}[1]{{\color{isseorange} #1}}
\newcommand{\hSecond}[1]{{\color{issegrey} #1}}

\newcounter{mathseed}
\setcounter{mathseed}{3}
\pgfmathsetseed{\arabic{mathseed}} % To have predictable results
% Define a background layer, in which the parchment shape is drawn
\pgfdeclarelayer{background}
\pgfsetlayers{background,main}


% This is the base for the fractal decoration. It takes a random point between the start and end, and
% raises it a random amount, thus transforming a segment into two, connected at that raised point
% This decoration can be applied again to each one of the resulting segments and so on, in a similar
% way of a Koch snowflake.
\pgfdeclaredecoration{irregular fractal line}{init}
{
  \state{init}[width=\pgfdecoratedinputsegmentremainingdistance]
  {
    \pgfpathlineto{\pgfpoint{random*\pgfdecoratedinputsegmentremainingdistance}{(random*\pgfdecorationsegmentamplitude-0.02)*\pgfdecoratedinputsegmentremainingdistance}}
    \pgfpathlineto{\pgfpoint{\pgfdecoratedinputsegmentremainingdistance}{0pt}}
  }
}


% define some styles
\tikzset{
   paper/.style={draw=black!10, blur shadow, every shadow/.style={opacity=1, black}, 
                 lower left=black!10, upper left=black!5, upper right=white, lower right=black!5, fill=none},
   irregular cloudy border/.style={decoration={irregular fractal line, amplitude=0.2},
           decorate,
     },
   irregular spiky border/.style={decoration={irregular fractal line, amplitude=-0.2},
           decorate,
     },
   ragged border/.style={ decoration={random steps, segment length=7mm, amplitude=2mm},
           decorate,
   }
}

\tikzset{
  normal border/.style={orange!30!black!10, decorate, 
     decoration={random steps, segment length=2.5cm, amplitude=.7mm}},
  torn border/.style={orange!30!black!5, decorate, 
     decoration={random steps, segment length=.5cm, amplitude=1.7mm}}}


\def\tornpaper#1{%
\ifthenelse{\isodd{\value{mathseed}}}{%
\tikz{
  \node[inner sep=1em] (A) {#1};  % Draw the text of the node
  \begin{pgfonlayer}{background}  % Draw the shape behind
  \fill[paper] % recursively decorate the bottom border
     \pgfextra{\pgfmathsetseed{\arabic{mathseed}}\addtocounter{mathseed}{1}}%
      {decorate[irregular cloudy border]{decorate{decorate{decorate{decorate[ragged border]{
        (A.north west) -- (A.north east)
      }}}}}}
      -- (A.south east)
     \pgfextra{\pgfmathsetseed{\arabic{mathseed}}}%
      {decorate[irregular spiky border]{decorate{decorate{decorate{decorate[ragged border]{
      -- (A.south west)
      }}}}}}
      -- (A.north west);
  \end{pgfonlayer}}
}{%
\tikz{
  \node[inner sep=1em] (A) {#1};  % Draw the text of the node
  \begin{pgfonlayer}{background}  % Draw the shape behind
  \fill[paper] % recursively decorate the bottom border
     \pgfextra{\pgfmathsetseed{\arabic{mathseed}}\addtocounter{mathseed}{1}}%
      {decorate[irregular spiky border]{decorate{decorate{decorate{decorate[ragged border]{
        (A.north east) -- (A.north west)
      }}}}}}
      -- (A.south west)
     \pgfextra{\pgfmathsetseed{\arabic{mathseed}}}%
      {decorate[irregular cloudy border]{decorate{decorate{decorate{decorate[ragged border]{
      -- (A.south east)
      }}}}}}
      -- (A.north east);
  \end{pgfonlayer}}
}}


% Macro to draw the shape behind the text, when it fits completly in the
% page
\def\parchmentframe#1{
\tikz{
  \node[inner sep=2em] (A) {#1};  % Draw the text of the node
  \begin{pgfonlayer}{background}  % Draw the shape behind
  \fill[normal border] 
        (A.south east) -- (A.south west) -- 
        (A.north west) -- (A.north east) -- cycle;
  \end{pgfonlayer}}}

% Macro to draw the shape, when the text will continue in next page
\def\parchmentframetop#1{
\tikz{
  \node[inner sep=2em] (A) {#1};    % Draw the text of the node
  \begin{pgfonlayer}{background}    
  \fill[normal border]              % Draw the ``complete shape'' behind
        (A.south east) -- (A.south west) -- 
        (A.north west) -- (A.north east) -- cycle;
  \fill[torn border]                % Add the torn lower border
        ($(A.south east)-(0,.2)$) -- ($(A.south west)-(0,.2)$) -- 
        ($(A.south west)+(0,.2)$) -- ($(A.south east)+(0,.2)$) -- cycle;
  \end{pgfonlayer}}}

% Macro to draw the shape, when the text continues from previous page
\def\parchmentframebottom#1{
\tikz{
  \node[inner sep=2em] (A) {#1};   % Draw the text of the node
  \begin{pgfonlayer}{background}   
  \fill[normal border]             % Draw the ``complete shape'' behind
        (A.south east) -- (A.south west) -- 
        (A.north west) -- (A.north east) -- cycle;
  \fill[torn border]               % Add the torn upper border
        ($(A.north east)-(0,.2)$) -- ($(A.north west)-(0,.2)$) -- 
        ($(A.north west)+(0,.2)$) -- ($(A.north east)+(0,.2)$) -- cycle;
  \end{pgfonlayer}}}

% Macro to draw the shape, when both the text continues from previous page
% and it will continue in next page
\def\parchmentframemiddle#1{
\tikz{
  \node[inner sep=2em] (A) {#1};   % Draw the text of the node
  \begin{pgfonlayer}{background}   
  \fill[normal border]             % Draw the ``complete shape'' behind
        (A.south east) -- (A.south west) -- 
        (A.north west) -- (A.north east) -- cycle;
  \fill[torn border]               % Add the torn lower border
        ($(A.south east)-(0,.2)$) -- ($(A.south west)-(0,.2)$) -- 
        ($(A.south west)+(0,.2)$) -- ($(A.south east)+(0,.2)$) -- cycle;
  \fill[torn border]               % Add the torn upper border
        ($(A.north east)-(0,.2)$) -- ($(A.north west)-(0,.2)$) -- 
        ($(A.north west)+(0,.2)$) -- ($(A.north east)+(0,.2)$) -- cycle;
  \end{pgfonlayer}}}

% Define the environment which puts the frame
% In this case, the environment also accepts an argument with an optional
% title (which defaults to ``Example'', which is typeset in a box overlaid
% on the top border
\newenvironment{parchment}[1][Example]{%
  \def\FrameCommand{\parchmentframe}%
  \def\FirstFrameCommand{\parchmentframetop}%
  \def\LastFrameCommand{\parchmentframebottom}%
  \def\MidFrameCommand{\parchmentframemiddle}%
  \vskip\baselineskip
  \MakeFramed {\FrameRestore}
  \noindent\tikz\node[inner sep=1ex, draw=black!20,fill=white, 
          anchor=west, overlay] at (0em, 2em) {\sffamily#1};\par}%
{\endMakeFramed}


\title{MiniBrass}
\author{Cost Function Networks}

\date{\today}

\begin{document}

\titleframe

%\begin{frame}
%\frametitle{Preferences in Constraint Solving}
%
%Constraint problem $(X, D, C)$ 
%\begin{itemize}
%  \item \cemph{Variables} $X$,
%\cemph{Domains} $D = (D_x)_{x \in X}$,
%\cemph{Constraints} $C$
%\end{itemize}
%
%\vspace*{1ex}
%
%How to deal with \cemph{over-constrained} problems?
%
%\vspace*{2ex}
%
%$((\{ \mathrm{x}, \mathrm{y}, \mathrm{z} \},
%\mathrm{D}_{\mathrm{x}} = \mathrm{D}_{\mathrm{y}} =
%\mathrm{D}_{\mathrm{z}} = \{ 1, 2, 3 \}), \{ \mathrm{c}_1,
%\mathrm{c}_2, \mathrm{c}_3 \})$ mit 
%\bgroup\abovedisplayskip4pt\belowdisplayskip4pt
%\begin{align*}
%  \mathrm{c}_1 &: \mathrm{x} + 1 = \mathrm{y}
%\\[-.4ex]
%  \mathrm{c}_2 &: \mathrm{z} = \mathrm{y} + 2
%\\[-.4ex]
%  \mathrm{c}_3 &: \mathrm{x} + \mathrm{y} \leq 3
%\end{align*}
%\egroup
%
%\begin{itemize}
%  \item Not all constraints can be satisfied simultaneously
%\begin{itemize} \pause
%  \item e.\,g., $\mathrm{c}_2$ forces $\mathrm{z} = 3$ and $\mathrm{y} = 1$, conflicting $\mathrm{c}_1$
%\end{itemize}
%
%  \item We can \cemph{choose} between assignments satisfying $\{ \mathrm{c}_1, \mathrm{c}_3 \}$ or $\{ \mathrm{c}_2, \mathrm{c}_3 \}$.
%\end{itemize}
%
%\vspace*{2ex}
%
%Which assignments $v \in [X \to D]$ should be \alert{preferred} by an agent/several agents?
%
%\end{frame}
%
%\begin{frame}
%\frametitle{Constraint Relationships}
%
%Approach~\cite{Schiendorfer13}
%\begin{itemize}
%  \item Define relation $R$ over constraints $C$ to denote which constraints are more important than others, e.\,g.
%\begin{itemize}
%  \item $\mathrm{c}_1$ is more important than  $\mathrm{c}_2$
%
%  \item $\mathrm{c}_1$ is more important than $\mathrm{c}_3$
%\end{itemize}
%\end{itemize}
%\begin{textblock*}{2.5cm}[1,1](\textwidth-1.5cm,\textheight-4.03cm)
%\begin{tikzpicture}[auto,
%                    ->,>=stealth',shorten >=1pt,thick,
%                    node distance=.7cm,inner sep=2pt,
%                    constraint/.style={circle,fill=black!15,draw,font=\sffamily\small}]
%\node[constraint node] (1) at (0, 0)                   {$\mathrm{c}_1$};
%\node[constraint node] (2) at ($ (1) + (-0.8, -0.8) $) {$\mathrm{c}_2$};  
%\node[constraint node] (3) at ($ (1) + ( 0.8, -0.8) $) {$\mathrm{c}_3$};  
%%  
%\path[every node/.style={font=\sffamily\tiny}]
%  (2) edge (1)
%  (3) edge (1)
%  ;
%\end{tikzpicture}
%\end{textblock*}
%
%\vspace*{5.6ex}
%
%Benefits
%\begin{itemize}
%  \item \cemph{Qualitative} formalism --- easy to specify
%  \item Graphical interpretation 
%\begin{itemize}
% \item Semantics (\alert{how} much more important is a constraint) regulated by 
%  \item \cemph{dominance properties} that are either ``hierarchical'' or ``egalitarian''
%  \item Single-Predecessors-Dominance (SPD) vs. Transitive-Predecessors-Dominance (TPD)
%\end{itemize}
%
%\end{itemize}
%
%%\vspace*{2ex}
%%\begin{small}
%%A.~Schiendorfer, J.-Ph.~Steghöfer, A.~Knapp, F.~Nafz, W.~Reif (2013)
%%\end{small}
%\end{frame}

\begin{frame}{Overview}
These few slides show how to use cost functions (also known as weighted constraints) in your model
such that a dedicated solver (\alert{toulbar2}) can access them


\vspace*{2ex}

To familiarize yourself with the basics, consider looking at:
\begin{itemize}
\item Step-by-Step enhancing a MiniZinc model (establishes the core elements)
\item Language Features 
\item Case Studies (for some specific examples)
\item Soft Global Constraints (\texttt{soft-alldiff}, \texttt{soft-regular}, \texttt{soft-gcc})
\end{itemize}

\vspace*{2ex}

\url{http://isse-augsburg.github.io/constraint-relationships/}
\end{frame}

\begin{frame}{Cost Functions}
\begin{itemize}
\item Different underlying model than pure constraint satisfaction and optimization~(CSOP)
\item Assumes a \emph{decomposable} cost function that represents the \alert{violation} of a constraint
\item Ranges from 0 (no violation) to some upper bound $k$ (maximally possible violation) 
\begin{itemize}
\item The objective of the CSOP is $F(x)$
\item i.e., for $x \in X$, $F(x) = \sum_{f \in F} f(x)$
\item Integer domains, integer objectives 
\end{itemize}
\vspace*{2ex}
\item Algorithmic support (\texttt{toulbar2}\footnote{\url{https://mulcyber.toulouse.inra.fr/projects/toulbar2/}} \cite{allouche2010toulbar2})
\begin{itemize}
\item Soft local consistency (specialized propagation)
\item Limited discrepancy search
\item Russian Doll Search, Branch-and-Bound etc
\end{itemize}
\end{itemize}
\end{frame}

\begin{frame}[fragile]{Example Cost Function}
Idea: Variables $X = \{x, y \}$, assign cost to every assignment (i.e., the cartesian product of the domains)
\begin{center}
\begin{tabular}{cc|c}
$x$ & $y$ & $c$ \\ 
\hline 
0 & 0 & 4 \\ 
0 & 1 & 3 \\ 
1 & 0 & 2 \\ 
1 & 1 & 4 \\ 
\end{tabular} 
\end{center}
\end{frame}

\begin{frame}[fragile]{Example Usage}
Assume you choose dinner and wine and let your friends 
rate dinner/wine combinations, as well as the wine individually
and you want to find the solution minimizing the overall \emph{dissatisfaction}.

\begin{lstlisting}
int: steak = 1; int: fish = 2; int: pizza = 3;
int: red = 1; int: white = 2;
set of int: MEAL = {steak, fish, pizza};
set of int: WINE = {red, white};
var MEAL: meal; var WINE: wine; 

var 0..4: mealA; var 0..4: mealB; var 0..4: mealC;
var 0..4: wineA; var 0..4: wineB; var 0..5: wineC;

include "soft_constraints/cost_functions.mzn";
\end{lstlisting}
%
%\begin{verbatim}
%----------
%meal = 2; % fish
%wine = 2; % white 
%\end{verbatim}
\end{frame}

\begin{frame}[fragile]{Example Usage}

\begin{lstlisting}
% Albert
constraint cost_function_binary(meal, wine, 
 [
   /* steak, red */       2,
   /* steak, white */     10,
   /* fish, red */        20,
   /* fish, white */      3,
   /* pizza, red */       5,
   /* pizza, white */     5
 ]
 , mealA);

constraint cost_function_unary(wine, [2, 1], wineA);
% analogously for Berta and Carl (see dinner.mzn)
\end{lstlisting}

\begin{verbatim}
----------
meal = 2; % fish
wine = 2; % white
\end{verbatim}
\end{frame}

\begin{frame}[fragile]{Native Support}
\begin{itemize}
\item Cost functions are \alert{first-class citizen} in \texttt{toulbar2}, accessed by \texttt{wcsp} files.
\item Classical constraints are cost functions that map to a value higher than some $k$ -- to denote unacceptable violation. 
\item Example:
\end{itemize}
\small
\begin{verbatim}
wcsp 2 2 1 100000000
2 2
2 0 1 0 4
0 0 4
0 1 3
1 0 2
1 1 4
\end{verbatim}
A simple WCSP file with $2$ variables, at most $2$ domain values, precisely $1$ cost function and an upper bound $k = 100000000$.\footnote{Read more about the WCSP format at \url{http://costfunction.org/mobyle/htdocs/portal/help/wcsp.html}} 
\end{frame}


\begin{frame}[fragile]{Access via Numberjack}
\begin{itemize}
\item Numberjack\footnote{\url{http://numberjack.ucc.ie/}}\cite{hebrard2010constraint} is a modeling package for multiple solvers, written in Python
\item Interfaces to toulbar2 using the \alert{objective} function!
\begin{itemize}
\item If the objective function (\texttt{Minimize}, \texttt{Maximize} objects) is a \emph{sum},
every sub-expression is translated to a cost function.
\item Very useful for \emph{weighted constraints}, in the sense that satisfaction maps to 0 and violation 
to a specific weight for all assignments of the variables in the scope of the cost function
\end{itemize}
\end{itemize}

\lstset{language=python}
\begin{lstlisting}
from Numberjack import * 
x,y,z = VarArray(3,0,2) # three variables, each with domain {0,1,2}
# state desirable constraints, and minimize their violation
# --> their negation being true
model = Model(
    Minimise( 2 * (x + 1 != y) +  # x + 1 = y should hold with weight 2
              (z != y + 2)   )) # z = y + 2 should hold with weight 1
solver.solve()
\end{lstlisting}

\lstset{language=mzn}

\end{frame}

\begin{frame}[fragile]{Generated WCSP}
\lstset{language=python}
\begin{lstlisting}
from Numberjack import * 
x,y,z = VarArray(3,0,2) # three variables, each with domain {0,1,2}
model = Model(
    Minimise( 2 * (x + 1 != y) +  # x + 1 = y should hold with weight 2
              (z != y + 2)   )) # z = y + 2 should hold with weight 1
solver.solve()
\end{lstlisting}

\lstset{language=mzn}

\small 
%\renewcommand{\ttdefault}{lmtt}
\begin{columns}[T]
    \begin{column}{.4\textwidth}
% Your image included here
\begin{semiverbatim}
wcsp 3 3 5 100000000
3 3 3 # order is z, y, x
2 0 1 0 9 # z = y + 2
0 0 1
0 1 1
0 2 1
1 0 1
1 1 1
1 2 1
\textcolor{isseorange}{2 0 0}
2 1 1
2 2 1
\end{semiverbatim}
    \end{column}
    \begin{column}{.4\textwidth}
\begin{Verbatim}[commandchars=\\\{\}]
2 1 2 0 9 # y = x + 1
0 0 2
0 1 2
0 2 2
\textcolor{isseorange}{1 0 0}
1 1 2
1 2 2
2 0 2
\textcolor{isseorange}{2 1 0}
\end{Verbatim}
\end{column}
\end{columns}
\end{frame}

\begin{frame}[fragile]{Accessing Toulbar2 via MiniZinc}

\lstset{language=python}
\begin{lstlisting}
from Numberjack import * 
x,y,z = VarArray(3,0,2) # three variables, each with domain {0,1,2}
model = Model(
    Minimise( 2 * (x + 1 != y) +  # x + 1 = y should hold with weight 2
              (z != y + 2)   )) # z = y + 2 should hold with weight 1
solver.solve()
\end{lstlisting}
The same model, written in MiniZinc\footnote{Compile using \texttt{mzn\_numberjack} with other solvers than \texttt{toulbar2} commented out in the portfolio solver \texttt{fzn/njportfolio.py} of the Numberjack source.}

\lstset{language=mzn}
\begin{lstlisting}
var 0..2: x; var 0..2: y; var 0..2: z;

var bool: sc1 = (x + 1 = y);
var bool: sc2 = (z = y + 2);

solve minimize 2 * (not sc1) + (not sc2); 
\end{lstlisting}
\end{frame}

\begin{frame}{Explicit Cost Function Modeling}
\begin{itemize}
\item Internally, the Cartesian product of the domains 
of the involved variables is built and an expression evaluated (e.g., $x + 1 = y$).
\item However, it is not possible to inject arbitrary cost functions from MiniZinc
\begin{itemize}
\item Enumerate support and cost function \emph{explicitly}
\item Use, e.g., MiniZinc functions and comprehensions to calculate cost vector
\end{itemize}
\end{itemize}
\begin{parchment}[Proposal]
\alert{Add cost functions as special global constraints to MiniZinc}
\end{parchment}
\end{frame}

\begin{frame}[fragile]{Decomposition in MiniZinc}
Provides a \emph{default implementation} of \texttt{cost\_function} that \emph{encodes}
cost functions using table constraints for any solver.
\small
\begin{lstlisting}
% e.g. x \in 0..2 ->, costs = [4, 1, 3] leads to 
% f(x -> 0) = 4, f(x -> 1) = 1, f(x -> 2) = 3
% costVariable takes the value of f(x)
predicate cost_function_unary(var int: x, 
                              array[int] of int: costs, 
                              var int: costVariable ) = 
let {
  array[int] of int: folded = [ x_ | x_ in dom(x)];
}
in ( cost_function_unary_safe(folded, costs, x, costVariable)); 
\end{lstlisting}
\end{frame}

\begin{frame}[fragile]{Safe Cost Function Encoding}
\small
\begin{lstlisting}
predicate cost_function_unary_safe(array[int] of int: folded, 
                                   array[int] of int: costs, 
                                   var int: x, var int: costVar) = 
assert(max(index_set(folded)) == max(index_set(costs)), 
"Dimensions of cost vector and flattened domains must agree", 
let {
   array[int] of int: tableDecomp = [ if(j == 1) then folded[i] 
                                      else  costs[i] endif 
                                      | i in index_set(costs), j in 1..2 ];
} in table([x,costVar], array2d(index_set(costs), 1..2, tableDecomp)));
\end{lstlisting}
Analogously (as of now) for \texttt{cost\_function\_binary} and \texttt{cost\_function\_ternary}.
All definitions are found in \texttt{soft\_constraints/cost\_functions.mzn}.
\end{frame}

\begin{frame}[fragile]{Example Usage}
\texttt{bincost-functions.mzn}
\begin{lstlisting}
array[1..2] of var 0..1: x; 
% default definitions from MiniBrass lib
include "soft_constraints/cost_functions.mzn";

var 0..10: cVar;
% 0, 0 -> 4; 0, 1 -> 3; 1, 0 -> 2; 1, 1 -> 4
constraint cost_function_binary(x[1], x[2], [4, 3, 2, 4], cVar);

solve minimize cVar;

\end{lstlisting}
\small
\begin{verbatim}
x = array1d(1..2 ,[0, 0]);
cVar = 4;
----------
x = array1d(1..2 ,[1, 0]);
cVar = 2;
----------
==========
\end{verbatim}
\end{frame}

\begin{frame}[fragile]{Usage with MiniZinc Functions}
\texttt{comprehensions.mzn}
\begin{lstlisting}
include "soft_constraints/cost_functions.mzn";

% shows an exemplary comprehension based cost function
var 0..1: x; var 0..1: y; var 0..10: costVar;
% x y | 4 - (x + y)   : (0,0) -> 4, (0,1) -> 3, (1, 0) -> 3, (1, 1) -> 2

function int: f(int: x, int: y) = (   4 - (x + y) );

constraint cost_function_binary(x, y, 
   [f(x_,y_) | x_ in dom(x), y_ in dom(y)], costVar);
solve minimize costVar;
\end{lstlisting}
\small
\begin{verbatim}
x = 1;
y = 1;
costVar = 2;
----------
==========
\end{verbatim}
\end{frame}

\begin{frame}[fragile]{Native Cost Function} \small
Toulbar2 supports \texttt{cost\_function} \emph{natively}. We should use this!\footnote{Experimentally
implemented in the forked Numberjack repository \url{https://github.com/Alexander-Schiendorfer/Numberjack} on the \texttt{feature/wcsp-encoding} branch.}

\vspace*{2ex}

Define \texttt{soft\_constraints/cost\_functions.mzn} in Numberjack's \texttt{mzn-lib} dir:
\begin{lstlisting}
predicate cost_function_binary(var int: x, var int: y, 
                               array[int] of int: costs, 
                               var int: costVariable );
\end{lstlisting}
such that \texttt{cost\_function\_binary} does not get decomposed but sent to Numberjack.

\vspace*{2ex}
On Numberjack's side, we need to treat \texttt{cost\_function\_binary} in \texttt{fzn2py}:
\lstset{language=python}
\begin{lstlisting}
def cost_function_binary(var1, var2, costs, costVar):
    return PostBinary(var1, var2, costs)
\end{lstlisting}
\lstset{language=mzn}
\texttt{costVar} is ignored since it is directly posted to the cost function.
\end{frame}

\begin{frame}[fragile]{How to use it?} \small
\begin{itemize}
\item Install MiniBrass from \url{http://isse-augsburg.github.io/constraint-relationships/}
\item Write models using \texttt{cost\_function}
\item Get the experimental Numberjack from \url{https://github.com/Alexander-Schiendorfer/Numberjack} (for now)
\item Try it!
\end{itemize}
\begin{lstlisting}
git clone -b feature/wcsp-encoding 
  https://github.com/Alexander-Schiendorfer/Numberjack.git 
\end{lstlisting}
\end{frame}

\begin{frame}[allowframebreaks]
        \frametitle{References}
        \bibliographystyle{apalike}
        \bibliography{references.bib,../common.bib}
\end{frame}


\end{document}

