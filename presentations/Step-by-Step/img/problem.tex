\tikzset{
    trajectory/.style={issegrey},
    emph/.style={isseorange},
    trajectorynode/.style={issegrey},
    demand/.style={MidnightBlue, thick},
    firstTraj/.style={ForestGreen},
    secTraj/.style={BrickRed}
} 
\begin{figure}
\begin{tikzpicture}[scale=1.0]
    % Draw axes
    \draw [<->,thick] (0,5) node (yaxis) [above] {$P(t)$}
        |- (8.5,0) node (xaxis) [right] {$t$};
        
    \node[overlay,text width=1.9cm, text centered, anchor=south, right] at (7.7,4.5)
    { \small 
    \begin{itemize} 
    \item[] { \color{MidnightBlue} \onslide<2->{\textbf{Demand}} } 
    \item[] { \color{ForestGreen} \onslide<3->{Plant $a$} } 
    \item[] { \color{BrickRed} \onslide<4->{Plant $b$} }  
    \item[] { \color{isseorange} \onslide<5->{\textbf{Supply}} }    
    \end{itemize}
    };        
       
	
%	\node[text width = 1.5cm ,text centered, anchor=west, right] at (2.5, 1)
%	{
%		$\mathbf{+}$
%	};
	
    %\node[text width=2.5cm, text centered, anchor=west, right] at (4,-.5)
    %{
    %		Kraftwerk $\mathsf{b}$
	%}; 
	
	%\node[text width = 1.5cm ,text centered, anchor=west, right] at (6.5, 1)
	%{
	%	$\mathbf{=}$
%	};
	%\node[text width=2.5cm, text centered, anchor=west, right] at (8,-.5)
    %{
    %		Demand
	%};      
    
     % draw second trajectory first graph 
     \onslide<2->{
    \draw[trajectory,demand] (0,3.9) coordinate (d20) -- (1,4.6) coordinate (d21);
    \draw[trajectory,demand] (d21) -- (2,4.4) coordinate (d22);
    \draw[trajectory,demand] (d22) -- (3,4.7) coordinate (d23);
    \draw[trajectory,demand] (d23) -- (4,3.5) coordinate (d24);
    \draw[trajectory,demand] (d24) -- (5,3.5) coordinate (d25);
    \draw[trajectory,demand] (d25) -- (6,3.5) coordinate (d26);
    \draw[trajectory,demand] (d26) -- (7,4.0) coordinate (d27);
    \draw[trajectory,demand] (d27) -- (8,4.5) coordinate (d28);
    
    % now for the circles
    \fill[trajectorynode,demand] (d21) circle (1pt);
    \fill[trajectorynode,demand] (d22) circle (1pt);
    \fill[trajectorynode,demand] (d23) circle (1pt);
    \fill[trajectorynode,demand] (d24) circle (1pt);    
    \fill[trajectorynode,demand] (d25) circle (1pt);
    \fill[trajectorynode,demand] (d26) circle (1pt);
    \fill[trajectorynode,demand] (d27) circle (1pt);
    \fill[trajectorynode,demand] (d28) circle (1pt);
    }
        
    \onslide<3->{
    % now for the first plant   
    \draw[trajectory,firstTraj] (0,1.9) coordinate (p10) -- (1,2.0) coordinate (p11);
    \draw[trajectory,firstTraj] (p11) -- (2,2.4) coordinate (p12);
    \draw[trajectory,firstTraj] (p12) -- (3,2.4) coordinate (p13);
    \draw[trajectory,firstTraj] (p13) -- (4,2.2) coordinate (p14);
    \draw[trajectory,firstTraj] (p14) -- (5,2.4) coordinate (p15);
    \draw[trajectory,firstTraj] (p15) -- (6,2.4) coordinate (p16);
    \draw[trajectory,firstTraj] (p16) -- (7,2.4) coordinate (p17);                   
    \draw[trajectory,firstTraj] (p17) -- (8,2.6) coordinate (p18);
    
    	\onslide<6>{
       \draw[trajectory,firstTraj,very thick] (p11) -- (p12);	
       \node[overlay,align=left,rectangle callout,%
             callout absolute pointer=(p11.west),xshift=-.5cm,yshift=-1.5cm,fill=isseorange!50] at (p12) {
            \scriptsize \textbf{Must} ramp up \\ \scriptsize due to inertia};
       
	}    
    
 	\onslide<8>{
       \draw[trajectory,firstTraj,very thick] (p15) -- (p16);	
       \draw[trajectory,firstTraj,very thick] (p16) -- (p17);
       
          \node[overlay,align=left,rectangle callout,%
             callout absolute pointer=(p16.north),xshift=-.5cm,yshift=0.55cm,fill=isseorange!50] at (p15) {
           \scriptsize  Wait 2 steps for \\ \scriptsize further ramp-up};
	} 
	
    % now for the circles of the first graph
    \fill[trajectorynode,firstTraj] (p11) circle (1pt);
    \fill[trajectorynode,firstTraj] (p12) circle (1pt);
    \fill[trajectorynode,firstTraj] (p13) circle (1pt);
    \fill[trajectorynode,firstTraj] (p14) circle (1pt);    
    \fill[trajectorynode,firstTraj] (p15) circle (1pt);
    \fill[trajectorynode,firstTraj] (p16) circle (1pt);
    \fill[trajectorynode,firstTraj] (p17) circle (1pt);
    \fill[trajectorynode,firstTraj] (p18) circle (1pt);        
    }
    
    \onslide<4->{
    % now for the second plant   
    \draw[trajectory,secTraj] (0,2.0) coordinate (p20) -- (1,2.6) coordinate (p21);
    \draw[trajectory,secTraj] (p21) -- (2,2.0) coordinate (p22);
    \draw[trajectory,secTraj] (p22) -- (3,2.2) coordinate (p23);
    \draw[trajectory,secTraj] (p23) -- (4,1.5) coordinate (p24);
    \draw[trajectory,secTraj] (p24) -- (5,1.4) coordinate (p25);
    \draw[trajectory,secTraj] (p25) -- (6,1.2) coordinate (p26);
    \draw[trajectory,secTraj] (p26) -- (7,1.6) coordinate (p27);                   
    \draw[trajectory,secTraj] (p27) -- (8,1.9) coordinate (p28);
	\onslide<6>{
       \draw[trajectory,secTraj,very thick] (p21) -- (p22);	
       \node[overlay,align=left,rectangle callout,%
             callout absolute pointer=(p21.north),xshift=+1cm,yshift=.5cm,fill=isseorange!50] at (p21) {
            \scriptsize Has to compensate};
	}    
	
	\onslide<7>{
       \draw[trajectory,secTraj,very thick] (p23) -- (p24);	
       \node[overlay,align=left,rectangle callout,%
             callout absolute pointer=(p24.south),xshift=+1cm,yshift=-.8cm,fill=isseorange!50] at (p24) {
             \scriptsize Cannot ramp down further};
	}    
    
     % now for the circles of the second graph
    \fill[trajectorynode,secTraj] (p21) circle (1pt);
    \fill[trajectorynode,secTraj] (p22) circle (1pt);
    \fill[trajectorynode,secTraj] (p23) circle (1pt);
    \fill[trajectorynode,secTraj] (p24) circle (1pt);    
    \fill[trajectorynode,secTraj] (p25) circle (1pt);
    \fill[trajectorynode,secTraj] (p26) circle (1pt);
    \fill[trajectorynode,secTraj] (p27) circle (1pt);
    \fill[trajectorynode,secTraj] (p28) circle (1pt);
    }
    
    \onslide<5->{
    % draw joint production first graph 
    \draw[trajectory,emph] (0,3.9) coordinate (s20) -- (1,4.6) coordinate (s21);
    \draw[trajectory,emph] (s21) -- (2,4.4) coordinate (s22);
    \draw[trajectory,emph] (s22) -- (3,4.6) coordinate (s23);
    \draw[trajectory,emph] (s23) -- (4,3.7) coordinate (s24);
    \draw[trajectory,emph] (s24) -- (5,3.8) coordinate (s25);
    \draw[trajectory,emph] (s25) -- (6,3.6) coordinate (s26);
    \draw[trajectory,emph] (s26) -- (7,4.0) coordinate (s27);
    \draw[trajectory,emph] (s27) -- (8,4.5) coordinate (s28);
    
	% now for the circles of the sum
    \fill[trajectorynode,emph] (s21) circle (1pt);
    \fill[trajectorynode,emph] (s22) circle (1pt);
    \fill[trajectorynode,emph] (s23) circle (1pt);
    \fill[trajectorynode,emph] (s24) circle (1pt);    
    \fill[trajectorynode,emph] (s25) circle (1pt);
    \fill[trajectorynode,emph] (s26) circle (1pt);
    \fill[trajectorynode,emph] (s27) circle (1pt);
    \fill[trajectorynode,emph] (s28) circle (1pt);
    }
    
	\node[text centered, anchor=north] at (1,0) { 1 }; \draw[thick] (1,0.05) -- (1,-.05);
	\node[text centered, anchor=north] at (2,0) { 2 }; \draw[thick] (2,0.05) -- (2,-.05);
	\node[text centered, anchor=north] at (3,0) { 3 }; \draw[thick] (3,0.05) -- (3,-.05);	
	\node[text centered, anchor=north] at (4,0) { 4 }; \draw[thick] (4,0.05) -- (4,-.05);
	\node[text centered, anchor=north] at (5,0) { 5 }; \draw[thick] (5,0.05) -- (5,-.05);
	\node[text centered, anchor=north] at (6,0) { 6 }; \draw[thick] (6,0.05) -- (6,-.05);
	\node[text centered, anchor=north] at (7,0) { 7 }; \draw[thick] (7,0.05) -- (7,-.05);	
	\node[text centered, anchor=north] at (8,0) { 8 }; \draw[thick] (8,0.05) -- (8,-.05);
	    

\end{tikzpicture}
\end{figure}  